\documentclass[11pt]{charter}

% El títulos de la memoria, se usa en la carátula y se puede usar el cualquier lugar del documento con el comando \ttitle
\titulo{Punto de cobro sin contacto para uso temporal de equipos.} 

% Nombre del posgrado, se usa en la carátula y se puede usar el cualquier lugar del documento con el comando \degreename
%\posgrado{Carrera de Especialización en Sistemas Embebidos} 
\posgrado{Carrera de Especialización en Internet de las Cosas} 
%\posgrado{Carrera de Especialización en Intelegencia Artificial}
%\posgrado{Maestría en Sistemas Embebidos} 
%\posgrado{Maestría en Internet de las cosas}

% Tu nombre, se puede usar el cualquier lugar del documento con el comando \authorname
\autor{Mariano Graziano} 

% El nombre del director y co-director, se puede usar el cualquier lugar del documento con el comando \supname y \cosupname y \pertesupname y \pertecosupname
\director{Nombre del Director}
\pertenenciaDirector{pertenencia} 
% FIXME:NO IMPLEMENTADO EL CODIRECTOR ni su pertenencia
\codirector{} % si queda vacio no se deberíá incluir 
\pertenenciaCoDirector{}

% Nombre del cliente, quien va a aprobar los resultados del proyecto, se puede usar con el comando \clientename y \empclientename
\cliente{Nombre del cliente}
\empresaCliente{Empresa del cliente}

% Nombre y pertenencia de los jurados, se pueden usar el cualquier lugar del documento con el comando \jurunoname, \jurdosname y \jurtresname y \perteunoname, \pertedosname y \pertetresname.
\juradoUno{Nombre y Apellido (1)}
\pertenenciaJurUno{pertenencia (1)} 
\juradoDos{Nombre y Apellido (2)}
\pertenenciaJurDos{pertenencia (2)}
\juradoTres{Nombre y Apellido (3)}
\pertenenciaJurTres{pertenencia (3)}
 
\fechaINICIO{25 de Agosto de 2020}		%Fecha de inicio de la cursada de GdP \fechaInicioName
\fechaFINALPlanificacion{13 de Octubre de 2020} 	%Fecha de final de cursada de GdP
\fechaFINALTrabajo{XX de Junio de 2021}		%Fecha de defensa pública del trabajo final


\begin{document}

\maketitle
\thispagestyle{empty}
\pagebreak


\thispagestyle{empty}
{\setlength{\parskip}{0pt}
\tableofcontents{}
}
\pagebreak


\section{Registros de cambios}
\label{sec:registro}


\begin{table}[ht]
\label{tab:registro}
\centering
\begin{tabularx}{\linewidth}{@{}|c|X|c|@{}}
\hline
\rowcolor[HTML]{C0C0C0} 
Revisión & \multicolumn{1}{c|}{\cellcolor[HTML]{C0C0C0}Detalles de los cambios realizados} & Fecha      \\ \hline
1.0      & Creación del documento y redaccion de puntos 1,2,3,4 y 6  & 06/09/2020 \\ \hline
%1.1      &                                                                 %& dd/mm/aaaa \\ \hline
%1.2      & Otro ejemplo \newline
%		   Con texto partido \newline
%		   En varias líneas \newline
%		   A propósito                                                     %& dd/mm/aaaa \\ \hline
\end{tabularx}
\end{table}

\pagebreak



\section{Acta de constitución del proyecto}
\label{sec:acta}

\begin{flushright}
Buenos Aires, \fechaInicioName
\end{flushright}

\vspace{2cm}

Por medio de la presente se acuerda con el Ing. \authorname\hspace{1px} que su Trabajo Final de la \degreename\hspace{1px} se titulará ``\ttitle'', consistirá esencialmente en el prototipo de un dispositivo de cobro para el uso de equipamiento electronico, y tendrá un presupuesto preliminar estimado de 600 hs de trabajo y \textcolor{red}{\$XXX}, con fecha de inicio \fechaInicioName\hspace{1px} y fecha de presentación pública \fechaFinalName.

Se adjunta a esta acta la planificación inicial.

\vfill

% Esta parte se construye sola con la información que hayan cargado en el preámbulo del documento y no debe modificarla
\begin{table}[ht]
\centering
\begin{tabular}{ccc}
\begin{tabular}[c]{@{}c@{}}Ariel Lutenberg \\ Director posgrado FIUBA\end{tabular} & \hspace{2cm} & \begin{tabular}[c]{@{}c@{}}\clientename \\ \empclientename \end{tabular} \vspace{2.5cm} \\ 
\multicolumn{3}{c}{\begin{tabular}[c]{@{}c@{}} \supname \\ Director del Trabajo Final\end{tabular}} \vspace{2.5cm} \\
%\begin{tabular}[c]{@{}c@{}}\jurunoname \\ Jurado del Trabajo Final\end{tabular}     &  & \begin{tabular}[c]{@{}c@{}}\jurdosname\\ Jurado del Trabajo Final\end{tabular}  \vspace{2.5cm}  \\
%\multicolumn{3}{c}{\begin{tabular}[c]{@{}c@{}} \jurtresname\\ Jurado del Trabajo Final\end{tabular}} \vspace{.5cm}                                                                     
\end{tabular}
\end{table}




\section{Descripción técnica-conceptual del proyecto a realizar}
\label{sec:descripcion}


El presente proyecto expresa el desafío de compatibilizar y articular todos los contenidos curriculares de la especialización con servicios de pago disponibles en el pais. 
El objetivo es crear un dispositivo que facilite al usuario de lavarropas comunitarios, una experiencia de pago sin fricciones. 

En la actualidad este tipo de equipamientos instalados en edificios o barrios cerrados, utilizan la modalidad de compra de fichas o monedas para habilitar el uso del mismo por un tiempo determinado. 
Las fichas insertan en un dispositivo-temporizador que reconoce el valor de las mismas y al cubrir un monto preestablecido habilita el uso del equipamiento por un tiempo determinado. 

Este equipamiento se administra de manera desatendida lo cual tiene las siguientes implicancias: 

\begin{itemize}
\item Cada uno de los dispositivos debe configurarse de modo presencial y periódicamente debe vaciarse el cofre de recaudación.

\item No hay forma remota de saber si el electrodomestico esta siendo usado por otro vecino.
\item Las fichas deben adquirirse en los kioskos de cercanía, que suelen desabastecerse casi inmediatamente en cuanto el proveedor las suministra.
\end{itemize}

Una de las principales motivaciones para llevar adelante este proyecto se produce en el contexto de la pandemia de SARS-COV-2. El aislamiento social y preventivo que está requiere, pone de manifiesto la obsolescencia de la practica en algo tan esencial como el lavado de ropa, de ir hasta comercios específicos y manipular dinero en efectivo.  

El presente proyecto se destaca por re definiry agilizar este proceso de pago enfocado en la experiencia del cliente. 
Como se puede ver en la Fig. 2. Se reemplaza la utilización de fichas y toda la logística que esto implica por una aplicacion movil o un medio RFID.

El usuario del equipo podra cargarle saldo por un medio de pago online (gateway de pago o transferencia PEI) u offline (corresponsalías extrabancarias como Rapipago o Pago Facil).

El dispositivo estará instalado próximo al equipo a utilizar y administrará el uso de la energía eléctrica del equipo.

Al querer utilizar el equipamiento, el usuario acerca su medio de pago al dispositivo el cual validara contra un servicio remoto que el medio de pago tenga saldo y el costo y plazo definido para ese dispositivo.

Se inicia una cuenta regresiva por el lapso contratado en la pantalla del dispositivo y la misma se replica en la aplicación multiplataforma del usuario. El dispositivo contará con un teclado numérico como interface de troubleshooting o uso alternativo a la tarjeta. 


Como complemento al desarrollo del dispositivo la innovación en este proyecto pasará por investigar la factibilidad técnica y económica de implementar en un proyecto de este tipo la la tecnología distribuida y potencial protocolo IOTA como medio de intercambio de información entre los distintos dispositivos intervinientes.
\vspace{25px}
\begin{figure}[htpb]
\centering 
\includegraphics[width=.7\textwidth]{./Figuras/diagBloques.png}
\caption{Diagrama en bloques del sistema}
\label{fig:diagBloques}
\end{figure}
\vspace{25px}


\newpage 
\newpage 

\section{Identificación y análisis de los interesados}
\label{sec:interesados}

\begin{consigna}{red} 
\begin{table}[ht]
%\caption{Identificación de los interesados}
%\label{tab:interesados}
\begin{tabularx}{\linewidth}{@{}|l|X|X|l|@{}}
\hline
\rowcolor[HTML]{C0C0C0} 
Rol           & Nombre y Apellido & Organización 	& Puesto 	\\ \hline
Auspiciante   &         -          &           -   	&     -   	\\ \hline
Responsable   & \authorname       & FIUBA        	& Alumno 	\\ \hline
Orientador    & \supname	      & \pertesupname 	& Director	Trabajo final \\ \hline
%Equipo        & miembro1 \newline 
%				miembro2          &              	&        	\\ \hline
%Opositores    &                   &              	&        	\\ \hline
Usuario final & Habitantes de complejos con area de Lavarropas Compartido                   & -             	&     -   	\\ \hline
\end{tabularx}
\end{table}


\end{consigna}

\newpage 

\section{1. Propósito del proyecto}
\label{sec:proposito}

 Este proyecto tiene tres propositos: 
 
 \begin{itemize}
\item El principal es digitalizar el medio de pago en servicios que aun utilizan fichas o cospeles.
\item En segundo lugar adquirir los conocimientos propios del ciclo de vida de un proyecto de IOT con servicios basados en la nube e integración con APIs desde su planificación hasta su ejecución.
\item Por ultimo es desarrollar y documentar en un contexto académico el análisis y potencial de la tecnología distribuida IOTA.

\end{itemize}
 
  

\section{2. Alcance del proyecto}
\label{sec:alcance}

El presente proyecto incluye: 
\begin{itemize}
		\item Diseño y armado de un dispositivo de cobro sin contacto.  
		\item Diseño y desarrollo de aplicación multiplataforma tanto para usuarios como para administradores del sistema. 
		\item Diseño de API para disponibilizar el acceso a terceros.
		\item Investigación sobre tecnologia IOTA y su impacto un proyecto de este estilo.
\end{itemize}		
 El presente proyecto no incluye:
\begin{itemize}	
		\item La integración con Plataformas de administradores de consorcios.
		\item La integración con entornos productivos de gateways de pago o Banca Online. 		
\end{itemize}		 
\section{3. Supuestos del proyecto}
\label{sec:supuestos}
Para el desarrollo del presente proyecto se supone que:
\begin{itemize}
\item Se podrá acceder a entornos sandbox de gateways de pago y banca online.
\item Todo el hardware necesario podra adquirirse en el país.
\item Dentro de los contenidos curriculares de la carrera es factible que se incluyan contenidos que permitan cubrir algunos de los requerimientos de manera mas eficiente a la propuesta en este documento. Los mismos se aplicaran en tanto el beneficio lo amerite y se pueda implementar la en mejora dentro de los plazos estipulados en el Acta de Constitución de este documento. 
\end{itemize}

\section{4. Requerimientos}
\label{sec:requerimientos}

\begin{enumerate}
\item Requerimientos funcionales del dispositivo:
	\begin{enumerate}
\item El punto de cobro debe estar conectado por uno o más medios a  internet de manera estable.
\item Si en el lugar físico (nodo de Punto de Cobros) hay más de un dispositivo uno de estos puede ser el gateway hacia internet de los restantes (prioridad baja) 
\item En caso de un fallo en la conexión debe poder garantizar al usuario el uso del lavarropas (cobro offline o lavado gratuito por desconexión)
\item El usuario debe poder pagar con un medio RFID o su teléfono celular sin tener contacto físico con el dispositivo y con la menor interacción posible.
\item En caso de corte de suministro eléctrico debe poder reiniciarse y volver al estado inmediato anterior al corte.
\item Deberá mostrar en la pantalla el estado del dispositivo y en caso de estar ocupado el tiempo restante.
\item Deberá contar con un teclado numérico como alternativa a usuarios que no quieran utilizar un medio electrónico de contacto y/o como interfaz de control en sitio asumiendo que es la única manera de tomar control del dispositivo.
\item Un dispositivo debe poder administrar más de un lavarropas (prioridad menor).
\item En caso de ser factible se deben poder usar medios rfid compatibles con los sistemas de control de accesos actuales.(prioridad menor)
\end{enumerate}
\item Requerimientos funcionales de la aplicación multiplataforma para el administrador del sistema punto de cobro:
	\begin{enumerate} 
\item Deberá poder gestionar la creación de nodos y la asignación de dispositivos a cada nodo. 
\item Deberá poder asignar usuarios administradores de nodos a un nodo y sus dispositivos.
	\end{enumerate}
\item Requerimientos funcionales de la aplicación multiplataforma para el administrador del nodo de lavarropas:
	\begin{enumerate} 
\item Deberá poder gestionar el costo y plazo de los lavarropas que están conectados a lavarropas.
\item Deberá poder gestionar el alta masiva y la forma de validación de los usuarios del nodo.
\item Deberá poder definir descuentos / promociones en función del nivel de usuario.
\item Deberá contar con un dashboard con el estado y niveles de uso.
\item Deberá poder definir los medios de cobro y de pago asociados al nodo. 
	\end{enumerate}
\item Requerimientos funcionales de la aplicación multiplataforma para el usuario final: 
\begin{enumerate} 
\item Deberá poder cargar saldo a su cuenta  con los medios de pagos habilitados.
\item Deberá poder gestionar los medios RFID asociados a su cuenta. 
\item Deberá poder ver la disponibilidad de los equipos a los que tiene acceso. 
\item Deberá poder transferir saldo a un tercero o a una tarjeta.
	\end{enumerate}
\item Requerimientos No funcionales.
\begin{enumerate}
\item Se debe loguear toda la informacion relevante a los fines de diagnosticar posibles problemas y/o mejoras al sistema.
\item Toda la información sensible del usuario debe almacenarse y transferirse de manera segura.
	\end{enumerate}
\end{enumerate}

\section{Historias de usuarios (\textit{Product backlog})}
\label{sec:backlog}

\begin{consigna}{red}
Descripción: En esta sección se deben incluir las historias de usuarios y su ponderación (\textit{history points}). Recordar que las historias de usuarios son descripciones cortas y simples de una característica contada desde la perspectiva de la persona que desea la nueva capacidad, generalmente un usuario o cliente del sistema. La ponderación es un número entero que representa el tamaño de la historia comparada con otras historias de similar tipo.
\end{consigna}

\section{5. Entregables principales del proyecto}
\label{sec:entregables}

\begin{itemize}
\item Prototipo funcional del dispositivo y de la aplicación a desarrollar.
\item Documento  de Viabilidad Económica Financiera.
\item Documento de Viabilidad Técnica / Funcional de la implementación de una red IOTA  en este proyecto.
\item Diagrama esquemático del sistema.
\item Manual de uso de la aplicación.
\item Informe de avance. 
\item Memoria del proyecto. 
\item Presentación ante el jurado.

\end{itemize}


\section{6. Desglose del trabajo en tareas}
\label{sec:wbs}
\begin{consigna}{red}
 En desarrollo.
 \end{consigna}
\begin{enumerate}
\item Tareas Preliminares
	\begin{enumerate}
	\item Planificación del Proyecto (30 hs)
	\item Análisis de Componentes del Dispositivo IOT ( 30 hs)
	\end{enumerate}
	\item Analisis de protocolos de comunicacion (30 hs)
\item Diseño del Sistema
	\begin{enumerate}
	\item Diseño de Base de Datos (tantas hs)
	\item Diseño de Servicios a Consumir (tantas hs)
	\item 
	\end{enumerate}
\item Desarrollo de la aplicación
	\begin{enumerate}
	\item Desarrollo  de FrontEnd Usuarios  (tantas hs)
	\item Desarrollo de FrontEnd Administradores (tantas hs)
	\item Integración con Servicios en Nube (tantas hs)
	\end{enumerate}
\item Desarrollo de la aplicacion IOT
	\begin{enumerate}
	\item Desarrollo de Modulo de Lectura RFID
	\item Desarrollo de Layout de Pantalla
	\item Tarea 3 (tantas hs)
	\end{enumerate}
\end{enumerate}

Cantidad total de horas: (tantas hs)


\section{7. Diagrama de Activity On Node}
\label{sec:AoN}

\begin{consigna}{red}
Armar el AoN a partir del WBS definido en la etapa anterior. 

%La figura \ref{fig:AoN} fue elaborada con el paquete latex tikz y pueden consultar la siguiente referencia \textit{online}:

%\url{https://www.overleaf.com/learn/latex/LaTeX_Graphics_using_TikZ:_A_Tutorial_for_Beginners_(Part_3)\%E2\%80\%94Creating_Flowcharts}

\end{consigna}

\begin{figure}[htpb]
\centering 
\includegraphics[width=.8\textwidth]{./Figuras/AoN.png}
\caption{Diagrama en \textit{Activity on Node}}
\label{fig:AoN}
\end{figure}

Indicar claramente en qué unidades están expresados los tiempos.
De ser necesario indicar los caminos semicríticos y analizar sus tiempos mediante un cuadro.
Es recomendable usar colores y un cuadro indicativo describiendo qué representa cada color, como se muestra en el siguiente ejemplo:



\section{8. Diagrama de Gantt}
\label{sec:gantt}

\begin{consigna}{red}
Utilizar el software Gantter for Google Drive o alguno similar para dibujar el diagrama de Gantt.

Existen muchos programas y recursos \textit{online} para hacer diagramas de gantt, entre las cuales destacamos:

\begin{itemize}
\item Planner
\item GanttProject
\item Trello + \textit{plugins}. En el siguiente link hay un tutorial oficial: \\ \url{https://blog.trello.com/es/diagrama-de-gantt-de-un-proyecto}
\item Creately, herramienta online colaborativa. \\\url{https://creately.com/diagram/example/ieb3p3ml/LaTeX}
\item Se puede hacer en latex con el paquete \textit{pgfgantt}\\ \url{http://ctan.dcc.uchile.cl/graphics/pgf/contrib/pgfgantt/pgfgantt.pdf}
\end{itemize}

Pegar acá una captura de pantalla del diagrama de Gantt, cuidando que la letra sea suficientemente grande como para ser legible. 
Si el diagrama queda demasiado ancho, se puede pegar primero la ``tabla'' del Gantt y luego pegar la parte del diagrama de barras del diagrama de Gantt.

Configurar el software para que en la parte de la tabla muestre los códigos del EDT (WBS).\\
Configurar el software para que al lado de cada barra muestre el nombre de cada tarea.\\
Revisar que la fecha de finalización coincida con lo indicado en el Acta Constitutiva.

En la figura \ref{fig:gantt}, se muestra un ejemplo de diagrama de gantt realizado con el paquete de \textit{pgfgantt}. En la plantilla pueden ver el código que lo genera y usarlo de base para construir el propio.

\begin{figure}[htbp]
\begin{center}
\begin{ganttchart}{1}{12}
  \gantttitle{2020}{12} \\
  \gantttitlelist{1,...,12}{1} \\
  \ganttgroup{Group 1}{1}{7} \\
  \ganttbar{Task 1}{1}{2} \\
  \ganttlinkedbar{Task 2}{3}{7} \ganttnewline
  \ganttmilestone{Milestone o hito}{7} \ganttnewline
  \ganttbar{Final Task}{8}{12}
  \ganttlink{elem2}{elem3}
  \ganttlink{elem3}{elem4}
\end{ganttchart}
\end{center}
\caption{Diagrama de gantt de ejemplo}
\label{fig:gantt}
\end{figure}

\end{consigna}

\section{9. Matriz de uso de recursos de materiales}
\label{sec:recursos}


\begin{table}
\label{tab:recursos}
\centering
\begin{tabularx}{\linewidth}{@{}|c|X|X|X|X|c|@{}}
\hline
\cellcolor[HTML]{C0C0C0} & \cellcolor[HTML]{C0C0C0} & \multicolumn{4}{c|}{\cellcolor[HTML]{C0C0C0}Recursos requeridos (horas)} \\ \cline{3-6} 
\multirow{-2}{*}{\cellcolor[HTML]{C0C0C0}\begin{tabular}[c]{@{}c@{}}Código\\ WBS\end{tabular}} & \multirow{-2}{*}{\cellcolor[HTML]{C0C0C0}\begin{tabular}[c]{@{}c@{}}Nombre \\ tarea\end{tabular}} & Material 1 & Material 2 & Material 3 & Material 4 \\ \hline
 &  &  &  &  &  \\ \hline
 &  &  &  &  &  \\ \hline
 &  &  &  &  &  \\ \hline
 &  &  &  &  &  \\ \hline
 &  &  &  &  &  \\ \hline
 &  &  &  &  &  \\ \hline
 &  &  &  &  &  \\ \hline
 &  &  &  &  &  \\ \hline 
 &  &  &  &  &  \\ \hline
 &  &  &  &  &  \\ \hline
 &  &  &  &  &  \\ \hline
 &  &  &  &  &  \\ \hline
 &  &  &  &  &  \\ \hline
 &  &  &  &  &  \\ \hline
 &  &  &  &  &  \\ \hline
 &  &  &  &  &  \\ \hline
 &  &  &  &  &  \\ \hline
 &  &  &  &  &  \\ \hline
 &  &  &  &  &  \\ \hline
 &  &  &  &  &  \\ \hline
 &  &  &  &  &  \\ \hline
 &  &  &  &  &  \\ \hline
 &  &  &  &  &  \\ \hline
 &  &  &  &  &  \\ \hline 
 &  &  &  &  &  \\ \hline
 &  &  &  &  &  \\ \hline
 &  &  &  &  &  \\ \hline
 &  &  &  &  &  \\ \hline

\end{tabularx}%
\end{table}


\section{10. Presupuesto detallado del proyecto}
\label{sec:presupuesto}

\begin{consigna}{red}
Si el proyecto es complejo entonces separarlo en partes:
\begin{itemize}
\item Un total global, indicando el subtotal acumulado por cada una de las áreas.
\item El desglose detallado del subtotal de cada una de las áreas.
\end{itemize}

IMPORTANTE: No olvidarse de considerar los COSTOS INDIRECTOS.

\end{consigna}

\begin{table}[htpb]
\centering
\begin{tabularx}{\linewidth}{@{}|X|c|r|r|@{}}
\hline
\rowcolor[HTML]{C0C0C0} 
\multicolumn{4}{|c|}{\cellcolor[HTML]{C0C0C0}COSTOS DIRECTOS} \\ \hline
\rowcolor[HTML]{C0C0C0} 
Descripción &
  \multicolumn{1}{c|}{\cellcolor[HTML]{C0C0C0}Cantidad} &
  \multicolumn{1}{c|}{\cellcolor[HTML]{C0C0C0}Valor unitario} &
  \multicolumn{1}{c|}{\cellcolor[HTML]{C0C0C0}Valor total} \\ \hline
 &
  \multicolumn{1}{c|}{} &
  \multicolumn{1}{c|}{} &
  \multicolumn{1}{c|}{} \\ \hline
 &
  \multicolumn{1}{c|}{} &
  \multicolumn{1}{c|}{} &
  \multicolumn{1}{c|}{} \\ \hline
\multicolumn{1}{|l|}{} &
   &
   &
   \\ \hline
\multicolumn{1}{|l|}{} &
   &
   &
   \\ \hline
\multicolumn{3}{|c|}{SUBTOTAL} &
  \multicolumn{1}{c|}{} \\ \hline
\rowcolor[HTML]{C0C0C0} 
\multicolumn{4}{|c|}{\cellcolor[HTML]{C0C0C0}COSTOS INDIRECTOS} \\ \hline
\rowcolor[HTML]{C0C0C0} 
Descripción &
  \multicolumn{1}{c|}{\cellcolor[HTML]{C0C0C0}Cantidad} &
  \multicolumn{1}{c|}{\cellcolor[HTML]{C0C0C0}Valor unitario} &
  \multicolumn{1}{c|}{\cellcolor[HTML]{C0C0C0}Valor total} \\ \hline
\multicolumn{1}{|l|}{} &
   &
   &
   \\ \hline
\multicolumn{1}{|l|}{} &
   &
   &
   \\ \hline
\multicolumn{1}{|l|}{} &
   &
   &
   \\ \hline
\multicolumn{3}{|c|}{SUBTOTAL} &
  \multicolumn{1}{c|}{} \\ \hline
\rowcolor[HTML]{C0C0C0}
\multicolumn{3}{|c|}{TOTAL} &
   \\ \hline
\end{tabularx}%
\end{table}


\section{11. Matriz de asignación de responsabilidades}
\label{sec:responsabilidades}
\begin{consigna}{red}
Establecer la matriz de asignación de responsabilidades y el manejo de la autoridad completando la siguiente tabla:

\begin{table}[htpb]
\centering
\resizebox{\textwidth}{!}{%
\begin{tabular}{|c|c|c|c|c|c|}
\hline
\rowcolor[HTML]{C0C0C0} 
\cellcolor[HTML]{C0C0C0} &
  \cellcolor[HTML]{C0C0C0} &
  \multicolumn{4}{c|}{\cellcolor[HTML]{C0C0C0}Listar todos los nombres y roles del proyecto} \\ \cline{3-6} 
\rowcolor[HTML]{C0C0C0} 
\cellcolor[HTML]{C0C0C0} &
  \cellcolor[HTML]{C0C0C0} &
  Responsable &
  Orientador &
  Equipo &
  Cliente \\ \cline{3-6} 
\rowcolor[HTML]{C0C0C0} 
\multirow{-3}{*}{\cellcolor[HTML]{C0C0C0}\begin{tabular}[c]{@{}c@{}}Código\\ WBS\end{tabular}} &
  \multirow{-3}{*}{\cellcolor[HTML]{C0C0C0}Nombre de la tarea} &
  \authorname &
  \supname &
  Nombre de alguien &
  \clientename \\ \hline
 &  &  &  &  &  \\ \hline
 &  &  &  &  &  \\ \hline
 &  &  &  &  &  \\ \hline
\end{tabular}%
}
\end{table}

{\footnotesize
Referencias:
\begin{itemize}
	\item P = Responsabilidad Primaria
	\item S = Responsabilidad Secundaria
	\item A = Aprobación
	\item I = Informado
	\item C = Consultado
\end{itemize}
} %footnotesize

Una de las columnas debe ser para el Director, ya que se supone que participará en el proyecto.
A su vez se debe cuidar que no queden muchas tareas seguidas sin ``A'' o ``I''.

Importante: es redundante poner ``I/A'' o ``I/C'', porque para aprobarlo o responder consultas primero la persona debe ser informada.

\end{consigna}

\section{12. Gestión de riesgos}
\label{sec:riesgos}

\begin{consigna}{red}
a) Identificación de los riesgos (al menos cinco) y estimación de sus consecuencias:
 
Riesgo 1: detallar el riesgo (riesgo es algo que si ocurre altera los planes previstos)
\begin{itemize}
\item Severidad (S): mientras más severo, más alto es el número (usar números del 1 al 10).\\
Justificar el motivo por el cual se asigna determinado número de severidad (S).
\item Probabilidad de ocurrencia (O): mientras más probable, más alto es el número (usar del 1 al 10).\\
Justificar el motivo por el cual se asigna determinado número de (O). 
\end{itemize}   

Riesgo 2:
\begin{itemize}
\item Severidad (S): 
\item Ocurrencia (O):
\end{itemize}

Riesgo 3:
\begin{itemize}
\item Severidad (S): 
\item Ocurrencia (O):
\end{itemize}


b) Tabla de gestión de riesgos:      (El RPN se calcula como RPN=SxO)

\begin{table}[htpb]
\centering
\begin{tabularx}{\linewidth}{@{}|X|c|c|c|c|c|c|@{}}
\hline
\rowcolor[HTML]{C0C0C0} 
Riesgo & S & O & RPN & S* & O* & RPN* \\ \hline
       &   &   &     &    &    &      \\ \hline
       &   &   &     &    &    &      \\ \hline
       &   &   &     &    &    &      \\ \hline
       &   &   &     &    &    &      \\ \hline
       &   &   &     &    &    &      \\ \hline
\end{tabularx}%
\end{table}

Criterio adoptado: 
Se tomarán medidas de mitigación en los riesgos cuyos números de RPN sean mayores a...

Nota: los valores marcados con (*) en la tabla corresponden luego de haber aplicado la mitigación.

c) Plan de mitigación de los riesgos que originalmente excedían el RPN máximo establecido:
 
Riesgo 1: plan de mitigación (si por el RPN fuera necesario elaborar un plan de mitigación).
  Nueva asignación de S y O, con su respectiva justificación:
  - Severidad (S): mientras más severo, más alto es el número (usar números del 1 al 10).
          Justificar el motivo por el cual se asigna determinado número de severidad (S).
  - Probabilidad de ocurrencia (O): mientras más probable, más alto es el número (usar del 1 al 10).
          Justificar el motivo por el cual se asigna determinado número de (O).

Riesgo 2: plan de mitigación (si por el RPN fuera necesario elaborar un plan de mitigación).
 
Riesgo 3: plan de mitigación (si por el RPN fuera necesario elaborar un plan de mitigación).

\end{consigna}


\section{13. Gestión de la calidad}
\label{sec:calidad}

\begin{consigna}{red}
Para cada uno de los requerimientos del proyecto indique:
\begin{itemize} 
\item Req \#1: copiar acá el requerimiento.

Verificación y validación:

\begin{itemize}
\item Verificación para confirmar si se cumplió con lo requerido antes de mostrar el sistema al cliente. Detallar 
\item Validación con el cliente para confirmar que está de acuerdo en que se cumplió con lo requerido. Detallar  
\end{itemize}

\end{itemize}

Tener en cuenta que en este contexto se pueden mencionar simulaciones, cálculos, revisión de hojas de datos, consulta con expertos, mediciones, etc.

\end{consigna}

\section{14. Comunicación del proyecto}
\label{sec:comunicaciones}

El plan de comunicación del proyecto es el siguiente:

\begin{table}[htpb]
\centering
\begin{tabularx}{\linewidth}{@{}|X|C{2.4cm}|C{3cm}|C{1.8cm}|C{2cm}|C{2.1cm}|@{}}
\hline
\rowcolor[HTML]{C0C0C0} 
\multicolumn{6}{|c|}{\cellcolor[HTML]{C0C0C0}PLAN DE COMUNICACIÓN DEL PROYECTO}           \\ \hline
\rowcolor[HTML]{C0C0C0} 
¿Qué comunicar? & Audiencia & Propósito & Frecuencia & Método de comunicac. & Responsable \\ \hline
                &           &           &            &                      &             \\ \hline
                &           &           &            &                      &             \\ \hline
                &           &           &            &                      &             \\ \hline
                &           &           &            &                      &             \\ \hline
                &           &           &            &                      &             \\ \hline
\end{tabularx}
\end{table}

\section{15. Gestión de compras}
\label{sec:compras}

\begin{consigna}{red}
En caso de tener que comprar elementos o contratar servicios:
a) Explique con qué criterios elegiría a un proveedor.
b) Redacte el Statement of Work correspondiente.
\end{consigna}

\section{16. Seguimiento y control}
\label{sec:seguimiento}

\begin{consigna}{red}
Para cada tarea del proyecto establecer la frecuencia y los indicadores con los se seguirá su avance y quién será el responsable de hacer dicho seguimiento y a quién debe comunicarse la situación (en concordancia con el Plan de Comunicación del proyecto).

El indicador de avance tiene que ser algo medible, mejor incluso si se puede medir en \% de avance. Por ejemplo,se pueden indicar en esta columna cosas como ``cantidad de conexiones ruteadeas'' o ``cantidad de funciones implementadas'', pero no algo genérico y ambiguo como ``\%'', porque el lector no sabe porcentaje de qué cosa.

\end{consigna}

\begin{longtable}{|m{1cm}|m{3.5cm}|m{2.2cm}|m{2cm}|m{3cm}|m{1.5cm}|}
\hline
\rowcolor[HTML]{C0C0C0} 
\multicolumn{6}{|c|}{\cellcolor[HTML]{C0C0C0}SEGUIMIENTO DE AVANCE}                                                                       \\ \hline
\rowcolor[HTML]{C0C0C0} 
Tarea del WBS 			& Indicador de avance & Frecuencia de reporte & Resp. de seguimiento & Persona a ser informada & Método de comunic. \\ \hline
\endfirsthead

\hline
\rowcolor[HTML]{C0C0C0} 
\multicolumn{6}{c}{\cellcolor[HTML]{C0C0C0}SEGUIMIENTO DE AVANCE}                                                                       \\ \hline
\rowcolor[HTML]{C0C0C0} 
Tarea del WBS 			& Indicador de avance & Frecuencia de reporte & Resp. de seguimiento & Persona a ser informada & Método de comunic. \\ \hline
\endhead

\multicolumn{6}{c}{Continúa}
\endfoot

\endlastfoot

1.1	& Fecha de inicio  & Única vez al comienzo & \authorname & \clientename, \supname & email \\ \hline
2.1	& Avance de las subtareas  & Mensual mientras dure la tarea & \authorname & \clientename, \supname & email \\ \hline

\end{longtable}

\begin{table}[!htpb]
\centering
%\begin{tabularx}{\linewidth}{@{}|X|X|X|X|X|X|@{}}
\begin{tabularx}{\linewidth}{@{}|X|C{2.5cm}|C{3cm}|C{2cm}|C{2cm}|C{2.5cm}|@{}}
\hline
\rowcolor[HTML]{C0C0C0} 
\multicolumn{6}{|c|}{\cellcolor[HTML]{C0C0C0}SEGUIMIENTO DE AVANCE}                                                                       \\ \hline
\rowcolor[HTML]{C0C0C0} 
Tarea del WBS & Indicador de avance & Frecuencia de reporte & Resp. de seguimiento & Persona a ser informada & Método de comunic. \\ \hline
 &  &  &  &  &  \\ \hline
 &  &  &  &  &  \\ \hline
 &  &  &  &  &  \\ \hline
 &  &  &  &  &  \\ \hline
 &  &  &  &  &  \\ \hline
\end{tabularx}%
%}
\end{table}

\section{17. Procesos de cierre}    
\label{sec:cierre}

\begin{consigna}{red}
Establecer las pautas de trabajo para realizar una reunión final de evaluación del proyecto, tal que contemple las siguientes actividades:

\begin{itemize}
\item Pautas de trabajo que se seguirán para analizar si se respetó el Plan de Proyecto original:
 - Indicar quién se ocupará de hacer esto y cuál será el procedimiento a aplicar. 
\item Identificación de las técnicas y procedimientos útiles e inútiles que se utilizaron, y los problemas que surgieron y cómo se solucionaron:
 - Indicar quién se ocupará de hacer esto y cuál será el procedimiento para dejar registro.
\item Indicar quién organizará el acto de agradecimiento a todos los interesados, y en especial al equipo de trabajo y colaboradores:
  - Indicar esto y quién financiará los gastos correspondientes.
\end{itemize}

\end{consigna}


\end{document}
